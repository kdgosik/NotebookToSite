\nonstopmode{}
\documentclass[letterpaper]{book}
\usepackage[times,inconsolata,hyper]{Rd}
\usepackage{makeidx}
\usepackage[utf8]{inputenc} % @SET ENCODING@
% \usepackage{graphicx} % @USE GRAPHICX@
\makeindex{}
\begin{document}
\chapter*{}
\begin{center}
{\textbf{\huge Package `NotebookToSite'}}
\par\bigskip{\large \today}
\end{center}
\begin{description}
\raggedright{}
\inputencoding{utf8}
\item[Type]\AsIs{Package}
\item[Title]\AsIs{Cretes a notebook and makes a site out of it.}
\item[Version]\AsIs{0.1.0}
\item[Author]\AsIs{Kirk Gosik}
\item[Maintainer]\AsIs{Kirk Gosik }\email{kgosik@gmail.com}\AsIs{}
\item[Description]\AsIs{Reproducible research package to call a function in order to automatically create directories
files that are common for a project. It both creates markdown files and also knits them into html files
to easily render into a static website.  This is set up to work with Github Pages placing the necessary
files into the docs directory of the project.}
\item[License]\AsIs{GPL-3}
\item[Encoding]\AsIs{UTF-8}
\item[LazyData]\AsIs{true}
\item[Imports]\AsIs{knitr, rmarkdown, utils,}
\item[RoxygenNote]\AsIs{6.0.1}
\item[NeedsCompilation]\AsIs{no}
\end{description}
\Rdcontents{\R{} topics documented:}
\inputencoding{utf8}
\HeaderA{create\_site}{create\_site}{create.Rul.site}
\keyword{knitr}{create\_site}
\keyword{purl}{create\_site}
%
\begin{Description}\relax
The function create\_site is meant to make reporducible workflows easier and more standardized.  It creates
many folders commonly used in projects suchs as docs, src, results and references.  It also creates an
Rmarkdown file for a notebook as well as uses knitr to spin it into a static website.  This html
file can be easily shared with collaborators or posted on github pages.  It is already placed in
the docs folder, which the user can enable a github pages website to be created from that folder
under the repository director.
\end{Description}
%
\begin{Usage}
\begin{verbatim}
create_site(github_username = "kdgosik")
\end{verbatim}
\end{Usage}
%
\begin{Arguments}
\begin{ldescription}
\item[\code{github\_username}] string of your github username to create a button to direct to your page.
\end{ldescription}
\end{Arguments}
%
\begin{Author}\relax
Kirk Gosik <kdgosik@gmail.com>
\end{Author}
%
\begin{SeeAlso}\relax
knitr
\end{SeeAlso}
\inputencoding{utf8}
\HeaderA{purl\_chunks}{purl\_chunks}{purl.Rul.chunks}
\keyword{knitr}{purl\_chunks}
\keyword{purl}{purl\_chunks}
%
\begin{Description}\relax
The function purl\_chunks takes an Rmarkdown file and returns each chunk as a
seperate R file.  It was found on stackoverflow at
(https://stackoverflow.com/questions/35855837/with-knitr-preserve-chunk-options-when-purling-chunks-into-separate-files)
\end{Description}
%
\begin{Usage}
\begin{verbatim}
purl_chunks(input_file, input_path = "src/chunks")
\end{verbatim}
\end{Usage}
%
\begin{Arguments}
\begin{ldescription}
\item[\code{input\_file}] String of the input file name to the Rmarkdown file that will be purlled into seperate chunks

\item[\code{input\_path}] String of the path where to place the R files of each of the chunks
\end{ldescription}
\end{Arguments}
%
\begin{Author}\relax
Kirk Gosik <kdgosik@gmail.com>
\end{Author}
%
\begin{SeeAlso}\relax
knitr
\end{SeeAlso}
\inputencoding{utf8}
\HeaderA{update\_site}{update\_site}{update.Rul.site}
\keyword{knitr}{update\_site}
\keyword{purl}{update\_site}
%
\begin{Description}\relax
The function update\_site is meant to make reporducible workflows easier and more standardized.  This will
update the R files that are in the src/chunks directory as well as knit new html files for the Notebook file.
\end{Description}
%
\begin{Usage}
\begin{verbatim}
update_site()
\end{verbatim}
\end{Usage}
%
\begin{Author}\relax
Kirk Gosik <kdgosik@gmail.com>
\end{Author}
%
\begin{SeeAlso}\relax
knitr
\end{SeeAlso}
\printindex{}
\end{document}
